\section{ Messung von Amplituden- und Phasengang der Filterschaltungen }



\subsection{Aufbau}

% grafik einbinden
\begin{figure}[H]
    \begin{center}
        \includegraphics[width=0.8\textwidth]{img/Blockschalt.PNG}
        \caption{Blockschaltbild, Quelle: "GNP3 Aktive RC-Filter.pdf" }
        \label{fig:A3_label}
    \end{center}
\end{figure}






\subsection{Amplitudengänge (mit Phasengängen) von Butterworth-, Tschebyscheff- und Bessel-Tiefpass gemeinsam in einem Plot über logarithmischer Frequenzachse, Frequenzbereich 100Hz...20kHz („AktFilt Ampl Phase.set“).}


\begin{table}[ht]
    \centering
    \begin{tabular}{|c|c|c|c|}\hline
    \tbf{Filter} & \tbf{Grenzfrequenz $f_g$} & \tbf{Phase $\SI{-60}{\degree}$}     &  \tbf{Phase $\SI{-120}{\degree}$}     &  \\ \hline
    Butterworth                   & \SI{1570.4}{\hertz} &     \SI{1069.2}{} &\SI{2417.2}{}     \\
    Tschebyscheff             & \SI{1582.6}{\hertz}  &   \SI{915.3}{}  &\SI{1425.2}{}    \\ 
    Bessel                &\SI{1582.6}{\hertz}   &\SI{1248.9}{}&\SI{3298.2}{}  \\ \hline
    \end{tabular}
    \caption{Filter Parameter}
\end{table}




\subsection{Phasengänge von Butterworth-, Tschebyscheff- und Bessel-Tiefpass gemeinsam in einem Plot über linearer Frequenzachse, Frequenzbereich 100Hz...4kHz. }


\begin{table}[ht]
    \centering
    \begin{tabular}{|c|c|c|c|}\hline
    \tbf{Filter} & \tbf{Grenzfrequenz $f_g$} & \tbf{Phase $\SI{-60}{\degree}$}     &  \tbf{Phase $\SI{-120}{\degree}$}     &  \\ \hline
    Butterworth                   & \SI{1584.6}{\hertz} &     \SI{1251.3}{} &\SI{3283.9}{}     \\
    Tschebyscheff             & \SI{1578.9}{\hertz}  &   \SI{914.6}{}  &\SI{1419.8}{}    \\ 
    Bessel                &\SI{1584.6}{\hertz}   &\SI{1066.4}{}&\SI{2404.5}{}  \\ \hline
    \end{tabular}
    \caption{Grenzfrequenzen der Filter}
\end{table}

% grafik einbinden
\begin{figure}[H]
    \begin{center}
        \includegraphics[width=0.8\textwidth, angle =-90]{img/3.2 Amplitudengang linear.png}
        \caption{Amplitudengang linear}
        \label{fig:A3_amp}
    \end{center}
\end{figure}
+

% grafik einbinden
\begin{figure}[H]
    \begin{center}
        \includegraphics[width=0.8\textwidth, angle =-90]{img/3.2 Phasengänge linear.png}
        \caption{Phasengänge linear}
        \label{fig:A3_phase}
    \end{center}
\end{figure}






\subsection{Amplitudengänge (mit Phasengängen) von Butterworth-, Tschebyscheff- und BesselHochpass gemeinsam in einem Plot über logarithmischer Frequenzachse, Frequenzbereich 100Hz...20kHz. }


\begin{table}[ht]
    \centering
    \begin{tabular}{|c|c|c|c|}\hline
    \tbf{Filter}     & \tbf{Grenzfrequenz $f_g$}  \\ \hline
    Butterworth                   & \SI{1632.5}{\hertz}            \\
    Tschebyscheff             & \SI{1607.3}{\hertz}           \\ 
    Bessel                &\SI{1619.8}{\hertz}     \\ \hline
    \end{tabular}
    \caption{Grenzfrequenzen der Filter}
\end{table}

% grafik einbinden
\begin{figure}[H]
    \begin{center}
        \includegraphics[width=0.8\textwidth, angle =-90]{img/3.3 Amplitudengänge HP log.png}
        \caption{\imgfilename}
        \label{fig:A3b_label}
    \end{center}
\end{figure}
% grafik einbinden
\begin{figure}[H]
    \begin{center}
        \includegraphics[width=0.8\textwidth, angle =-90]{img/3.3 Phasengänge HP log.png}
        \caption{\imgfilename}
        \label{fig:A3c_label}
    \end{center}
\end{figure}





\subsection{Amplitudengänge (mit Phasengängen) von Bandpass und Bandsperre gemeinsam in einem Plot über logarithmischer Frequenzachse, Frequenzbereich 1kHz...2,5kHz. }
 

%multi figure
\begin{figure}[H]
\begin{center}
\subfloat[Amplitudengang Bandpass]{\includegraphics[width = \textwidth/3, angle =-90]{img/3.4 Amplitudengang Bandpass.png}}  
\subfloat[Amplitudengang Bandsperre]{\includegraphics[width = \textwidth/3, angle =-90]{img/3.4 Amplitudengang Bandsperre.png}}\\
\subfloat[Bandbreite Bandpass]{\includegraphics[width = \textwidth/3, angle =-90]{img/3.4 Bandbreite Bandpass.png}} 
\subfloat[Bandbreite Bandsperre]{\includegraphics[width = \textwidth/3, angle =-90]{img/3.4 Bandbreite Bandsperre.png}} 
\caption{Bandbreiten}
\label{fig:A3_mult}
\end{center}
\end{figure}


\begin{table}[ht]
    \centering
    \begin{tabular}{|c|c|c|c|}\hline
    \tbf{Filter}     & \tbf{Mitte} & \tbf{$\SI{-3}{\decibel}$ unten} & \tbf{$\SI{-3}{\decibel}$ oben} \\ \hline
    Bandpass                   & \SI{1932.3}{\hertz}  & \SI{1759.1}{\hertz} & \SI{2085.9}{\hertz}\\
    Bandsperre                 & \SI{1914.2}{\hertz}  & \SI{1912}{\hertz} & \SI{1916.3}{\hertz}\\ \hline
    \end{tabular}
    \caption{Beispiel Addieren:}
\end{table}






%Amplitudengänge von Butterworth-, Tschebyscheff- und Bessel-Tiefpass und Markierung der Grenzfrequenzen Phasengänge von Butterworth-, Tschebyscheff- und Bessel-Tiefpass und Markierung der Frequenzen für eine Phasenverschiebung von -60° und -120° Amplitudengänge von Butterworth-, Tschebyscheff- und Bessel-Hochpass und Markierung der Grenzfrequenzen Amplitudengänge von Bandpass und Bandsperre und Markierung der Grenzfrequenzen Auswertung: Alle Messwerte der Grenzfrequenzen sind in einer Tabelle mit den Vorgabewerten einzutragen und zu vergleichen; Aus den Frequenzen bei 60° und 120° Phasenverschiebung sind die Koeffizienten a1, b1 der 3 Tiefpässe zu berechnen und mit den Vorgabewerten zu vergleichen 





















\subsection{Auswertung}
%%%%%%%%%%%%%%%%%%%%%%%%%%%%%%%%%%%%%%%%%%%%%%%%Comments
%Amplitudengänge von Butterworth-, Tschebyscheff- und Bessel-Tiefpass und Markierung der Grenzfrequenzen Phasengänge von Butterworth-, Tschebyscheff- und Bessel-Tiefpass und Markierung der Frequenzen für eine Phasenverschiebung von -60° und -120° Amplitudengänge von Butterworth-, Tschebyscheff- und Bessel-Hochpass und Markierung der Grenzfrequenzen Amplitudengänge von Bandpass und Bandsperre und Markierung der Grenzfrequenzen Auswertung: Alle Messwerte der Grenzfrequenzen sind in einer Tabelle mit den Vorgabewerten einzutragen und zu vergleichen; Aus den Frequenzen bei 60° und 120° Phasenverschiebung sind die Koeffizienten a1, b1 der 3 Tiefpässe zu berechnen und mit den Vorgabewerten zu vergleichen