\section{ Vorausberechnung der Schaltung}

Folgende Eigenschaften der Filter sind in häuslicher Vorarbeit zu bestimmen und zum Praktikumstermin vorzulegen:
\subsectio{Grundverstärkung und Grenzfrequenzen der Hoch- und Tiefpässe }
\subsectio{Mittenfrequenz und Bandbreite des Bandpasses }
\subsectio{Sperrfrequenz der Bandsperre }
\subsectio{Übertragungsfunktionen mit Matlab symbolisch berechnet }
\subsectio{Übertragungsfunktionen (Betrag und Phase) mit Matlab numerisch berechnet} 
\subsectio{Spice Simulationsergebnisse für das Frequenzverhalten (AC-Analyse)}

Die Eigenschaften 1-3) können entweder berechnet (siehe Abschnitte 6 und 7) oder aus Simulationen mit Spice (z.B. PSpice oder LTSpice) entnommen werden. Der Berechnungsgang bzw. Schaltung und Plot der Spice-Simulation sind zum Labortermin vorzulegen und in das Protokoll mit aufzunehmen, genauso wie der Matlab-Code.


\subsection{Aufbau}


%%%%%%%%%%%%%%%%%%%%%%%%%%%%%%%%%%%%%%%%%%%%%%%%Comments
%zu 2 :Formeln bzw. Spice-Plots und Ergebnisse der Vorausberechnung für Grundverstärkung und Grenzfrequenzen der Hoch- und Tiefpässe, Mittenfrequenz und Bandbreite des Bandpasses, Sperrfrequenz der Bandsperre. 